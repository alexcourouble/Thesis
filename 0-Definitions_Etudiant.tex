%% -----------------------------------
%% ---> A MODIFIER PAR L'ETUDIANT <---
%% -----------------------------------
%%
%% Commandes qui affichent le titre du document, le nom de l'auteur, etc.
\newcommand\monTitre{ON HISTORY-AWARE MULTI-ACTIVITY EXPERTISE MODELS}
\newcommand\monPrenom{Alexandre}
\newcommand\monNom{Courouble}
\newcommand\monDepartement{génie informatique et génie logiciel}
\newcommand\maDiscipline{génie informatique}
\newcommand\monDiplome{M}        % (M)aîtrise ou (D)octorat
\newcommand\anneeDepot{2018}
\newcommand\moisDepot{Mars}
\newcommand\monSexe{M}           % "M" ou "F"
\newcommand\PageGarde{N}         % "O" ou "N"
\newcommand\AnnexesPresentes{N}  % "O" ou "N". Indique si le document comprend des annexes.
\newcommand\mesMotsClef{Software Engineering, Mining Software Repositories, Software Expertise}
%%
%%  DEFINITION DU JURY
%%
%%  Pour la définition du jury, les macros suivantes sont definies:
%%  \PresidentJury, \DirecteurRecherche, \CoDirecteurRecherche, \MembreJury, \MembreExterneJury
%%
%%  Toutes les macros prennent 4 paramètres: Sexe (M/F), Prénom, Nom, Titres
\newcommand\monJury{\PresidentJury{M}{Giulio}{Antoniol}{Ph.~D.}\\
\DirecteurRecherche{M}{Bram}{Adams}{Doctorat}\\
\MembreJury{M}{Ettore}{Merlo}{Ph.~D.}}
