\Chapter{CRITICAL LITERATURE REVIEW}\label{sec:RevLitt}


Our research project touches upon two areas of software engineering research: \textit{mining software repositories} and \textit{software engineering expertise}. Besides these two research fields, we relied on another type of literature to conduct this research project: Git and Linux documentation. This chapter provides a critical literature review of the our two academic research areas and a description of the information available in the Linux and Git documentation, as well as how it helped us finding solutions for the problems encoutered.



\section{Mining Software Repositories}

In addition to providing a contribution platform, \ac{SCM} systems track and save large amounts of information about each changes brought to the source code. During the lifetime of the project, the \ac{SCM} acquires a large amount of data about the development of the project. Mining software repositories researchers \textit{mine} this data for their research projects. 

Software repositories are not limited to \ac{SCM}. There are other entities present in software repositories that research mine to gather information about software projects. These entities include bug tracking systems, mailing lists, source code, and issue tracking systems. Over the years, researchers have used mining software reporistories techniques that enabled them to research different topics of software engineering~\citep{Bird-2009}.

In the scope of our research, we used mining software repositories techniques in each different part of the project. Chapter 3 discusses two open source projects we created during this research project. The data used for both projects came from mining the Linux Kernel repository. We eventually used this data for the creation of our expertise model.

One of the difficulty often encountered by researchers in mining software repositories is the inability to link data coming from different entities of the software repository. In the case of the linux kernel, the dificulty was to link data from the mailing lists to the data from the git repository. A dificulty we addressed with an algorithm in \citep{jiang14}. 

Furthermore, \citep{armstrong} studied the difference between \textit{unicast} and \textit{broadcast} review systems. A unicast review system, like gerrit, provides an environment in which the code reviews are only visible by the author of the patch. On the contrary, broadcast review system, like the email system used by the Linux community, shows the code reviews to each reader of the mailing list. There are advantages and inconvinients to both systems. The authors note that unicast reviews lead to less bugs in the future, but that broadcast systems allows for faster review cycles and allow beginers to learn the code base faster. 



\section{Software Engineering Expertise}
\label{sec:expertise_models}

Many different studies explored the concept of expertise in software engineering. the authors of two early studies, Expertise Browser \citep{mockus02} and Expertise Recommender\citep{McDonald}, expressed the importance of understanding developers' expertise level. McDonald et al. approach the topic from a problem solving percpective. Today's developers have many different resources at their disposal for the purpose of problem solving. Stack Overflow\footnote{\url{https://stackoverflow.com/}}, a programming question answer exchange, is used by developers from around the world that are lookiing for solutions for complex problems. Before Stack Overflow's creation in 2008, developers did not have easy access to a large database of solutions provided by \textit{experts} from various topics. The authors of the Expertise Recommender were seeking to solve this problem by providing an architecture capable of recommending experts for given parts of the software project, for the sake of finding an expert to solve an issue. \citep{mockus02} approach the issue differently. They provide and expertise model to solve the issue of replacing or adding new expert to a distributed software engineering project. They argue that the tool would reduce the time lost by engineers attempting to find a new expert for their team. 


Each of these previous studies ~\citep{Bhattacharya}, ~\citep{mockus02}, ~\citep{McDonald}, and~\citep{Fritz-2007} base their measures of expertise among developers on the tacit assumption that experience is acquired through development activities, such as number of lines of code contributes or the number of commits authored. The author in \citep{Fritz-2007} examined the reliability of this assumption. Through a review of the many studies in psychology on knowledge and expertise, \citep{Fritz-2007} disovered that there was no sufficient evidence that activity does determine one's knowledge. The authors conducted a quantitative study on 19 java programmer to assess the accuracy of these finding. With this survey, they discovered that multiple activity related heuristics influenced developers' knowledge. These heurstics include authorship, role, work experience, and activity, which confirms the suitablility of the metrics used in previous work (LOC and commits). Furthermore,~\citep{Fritz-2007} mention the effect of code stability to code knolwedge. The authors emphasized the importance of the notion of history in determining knowledge.  

Globally, we found that the previous expertise models on the topic fail to address several important activities present in software development mentioned in~\citep{Fritz-2007}.Even though the authors give insightful advice on how developers acquire knoledge, they do not provide an expertise model capable of recommending expert for a certain code area. 


\section{Open Source Participation}

Previous work studies developers' motivation in \ac{OSS}. In~\citep{Wu-oss}, warn that the losely organized nature of OSS development could be associated with a high turnover rate and in unexpected departures. Other work ~\citep{Rigby} studies the impact of a high turn over on the organization. The authors argue that departing developers bring the amount of knowledge they acquired during their time as a contributor. We believe that this implies that OSS projects are at risk of knoledge loss and we believe that accurate expertise modeling could assist in addressing this issue.


\alex{Not sure of which other research areas using OSS that is relevant to our research}


