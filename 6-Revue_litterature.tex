\Chapter{CRITICAL LITERATURE REVIEW}\label{sec:RevLitt}


Our research project touches upon two areas of software engineering research: \textit{mining software repositories} and \textit{software engineering expertise}. One could argue that the field of mining software repositories enabled the creation of expertise models, as these models are based on metrics only available through the use of mining software repositories techniques. Besides these two research fields, we relied on another type of literature to conduct this research project: Git and Linux documentation. This chapter provides a critical literature review of the our two academic research areas and a description of the information available in the Linux and Git documentation, as well as how it helped us finding solutions for the problems encoutered.



\section{Mining Software Repositories}

To answer the difficulty associated with collaboration in large software projects, tools were created to ease collaboration between team members were created. In addition to providing a contribution platform, \ac{SCM} systems track and save large amounts of information about each changes brought to the source code. During the lifetime of the project, the \ac{SCM} acquires a large amount of data about the development of the project. Mining software repositories researchers \textit{mine} this data for their research projects. 

Software repositories are not limited to \ac{SCM}. There are other entities present in software repositories that research mine to gather information about software projects. These entities include bug tracking systems, mailing lists, source code, and issue tracking systems. Over the years, researchers have used mining software reporistories techniques that enabled them to research different topics of software engineering. \alex{cite daniel's git paper}

In the scope of our research, we used mining software repositories techniques in each different part of the project. The data used for both open source project created during this project came from mining the Linux Kernel repository. We eventually used this data for the creation of our expertise model.

One of the difficulty often encourntered by researchers in mining software repositories is the inability to link data coming from different entities of the software repository. In the case of the linux kernel, the dificulty was to link data from the mailing lists to the data from the git repository. A dificulty we addressed with Email2git.



\section{Software Engineering Expertise}
\label{sec:expertise_models}




\section{Documentation}




\subsection{Linux Documentation}

Start with the size , complexity, and success of the linux. As software and as a community. The software side of the success is partly due to 
cite intro \autoref{sec:contrib-process}


\subsection{Git Documentation}
