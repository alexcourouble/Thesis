\Chapter{EMAIL2GIT: FROM ACADEMIC RESEARCH TO OPEN-SOURCE SOFTWARE}\label{sec:Theme2}


\section{Previous Publications and Original Algorithm}

The original algorithm capable of backtracking patches from commits was introduced in two papers~\citep{msr13jojo,jiang14} published by Jiang, a former member of the MCIS Lab. Originally written in Perl, the script was a great proof of concept. The general idea of the script was to compare the +/- lines from both the git commits and the email patches. A match was found if the proportion of identical +/- lines was above a certain threshold. Although this script was a great proof of concept, it had difficulties scalling to 8 years of emails and commits. 



\section{Scalling the Algorithm}

Because we wanted Email2git to be a usable and practical tool, we needed a way to display the patches and the code reviews in a browser. Fortunatly, a great existing open-source tool called \textbf{Patchwork}\footnote{\url{https://github.com/getpatchwork/patchwork}} perfectly answers our requirements. Patchwork is a tool designed to assist maintainers of open source projects using an email-based contribution process. It tracks the mailing lists used by developers to submit patches and recieve code reviews. The tool extracts each detected patch as well as its associated reviews, then displays them in a web-based user interface. 

We were granted read access to the MySQL database behind a patchwork instance hosted on kernel.org\footnote{\url{https://patchwork.kernel.org/}}. This instance has been tracking 69 of the many linux subsystems mailing lists since 2009, giving us the oportunity to analyse over \textit{1.4 million} patches.

In addition to being a great data source, patchwork.kernel.org is also a great way for us to display the patches and the code reviews associated with commits to the users. The only limitation of this patchwork instance is that it does not track some major mailing lists, particularly some of the \texttt{Net} mailing lists. 


Since we had access to email patches dating back to 2009, we decided to extract git commits from the Linux git repository from the same date, which represent over \textit{500,000 commits} to analyse. Unfortunatly, this amount of data was too large for the orignal algorithm to parse in a timely fashion, which called for a new, scalable algorithm that applies exploits mentioned in \citep{msr13jojo,jiang14}.




\section{The Data}

There are two sides to this matching process: the Linux git repository and the archives containing the patches sent in mailing lists over the years. We need to extract data from both sides. 

\subsection{Extracting the Commits}


\subsection{Extracting the Patches}








\section{Serving the Matches}





