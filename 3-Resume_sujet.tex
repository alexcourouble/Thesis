% Résumé du mémoire.
%
%   Le résumé est un bref exposé du sujet traité, des objectifs visés,
% des hypothèses émises, des méthodes expérimentales utilisées et de
% l'analyse des résultats obtenus. On y présente également les
% principales conclusions de la recherche ainsi que ses applications
% éventuelles. En général, un résumé ne dépasse pas quatre pages.
%
%   Le résumé doit donner une idée exacte du contenu du mémoire ou de la thèse. Ce ne
% peut pas être une simple énumération des parties du document, car il
% doit faire ressortir l'originalité de la recherche, son aspect
% créatif et sa contribution au développement de la technologie ou à
% l'avancement des connaissances en génie et en sciences appliquées.
% Un résumé ne doit jamais comporter de références ou de figures.

\chapter*{RÉSUMÉ}\thispagestyle{headings}
\addcontentsline{toc}{compteur}{RÉSUMÉ}

Durant l'évolution d'un projet de logiciel, les contributions individuelles d'un developeur present dans le projet vont lentement se faire remplacer par les contributions d'autre dévelopeurs. Ceci engendrera l'érosion de l'empreinte des contributions de ce developeur. Bien que les connaissances de ce dévelopeur n'ont pas disparu du jour au lendemain, pour une personne externe au projet, l'expertise de ce developeur est devenue invisible. 

Grace à une étude empirique sur une periode de 5 années de developement de Linux, nous étudions le phénomène de l'érosion de l'expertise en créant un modèle bidimentionnel. La première dimention de notre modèle prend en compte les différentes activités entreprises par les membres de la communauté de développement de Linux, comme les contributions en termes de code, les contributions aux revues de code soumit par d'autre dévelopeurs, ou encore la soumission de code d'autres dévelopeurs en amont. La deuxiéme dimention de notre modèle prend en compte l'historique des contributions citées plus haut pour chaque dévelopeurs. 

En applicant ce modèle, nous decouvrons que, bien que les empreintes de contributions de certain dévelopeurs diminuent avec le temps, leurs expertise survit grace à leurs implications dans les divereses activités mentionées plus haut. 

% Dans ce mémoire, nous expliquons les differentes étapes entreprises pour la création de ce modèle. Premiérement, nous fournissons

% \alex{more}