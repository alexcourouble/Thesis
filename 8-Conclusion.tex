\Chapter{CONCLUSION}
\label{sec:Conclusion}


The research project described in this thesis establishes a history-aware model capable of predicting experts in a large software project. We describe the different metrics acquired for the creation of the model as well as the techniques used to mine the necessary data. We also provide an empirical evaluation of the model based on a list of known Linux experts. We were able to make some of the data available to the Linux community through two open source tool available online. We describe the different steps undertaken during the deployment of those tools. We also provide an evaluation of our expertise model.


\section{Advancement of Knowledge}


In addition to communicating our findings regarding expertise models in a submitted scholarly article, we were able to build two interfaces to further share our data with the rest of the Linux community. The main advancement of knowledge carried by our research project is the new dimentions used in our expertise model. State of the art techniques fail to include other aspects of software engineering, like code reviews and upstream committing. Our model is more appropriate to large organization where mainainers or managers are usually too busy to continue contributing code to the project. 

\section{Limits, Constraints, and Recommendations}
% \label{sec:Limits and constraints}






% \subsection{Paper}

There are two main limitations to the model proposed in our submitted paper. The first limitation is a direct concequence of Email2git's limitation. The missing mailing lists cause an uneven distribution of the matched commits accross the different subdirectories of the kernel. To address this limitation, we only studied subsystems with a certain percentage of matched commits. This ensured homogeneity of the data among the different subsystems. 

The other limitation is the validation technique we implement to assess the performance of our model. We use the maintainers currently active in the subsystem for the studied release. The issue with the technique is that since our model is partially based on activities usually related to maintainership, such as code reviews and upstream committing. To evaluate our model as \textbf{maintainer recommender} instead of an \textbf{expert recommender}, we would have to look at the developers that were selected as a replacement for a departing maintainer. A strong model would be capable of detecting the chosen developer for the release before the maintainer change.

At last, we cite the existing link between code stability and knowldge of that code area. Since our model offers a customaizable historical weight function, we could choose this weight function according to the stability of the code. For example, we could implement an exponential weight function to a very stable code base, as older contributions should account for more of the expertise measure. Furthermore, a code base undergoing large amounts of changes could require a logarithmic weight function, as older contributions should not affect current expertise as much as recent contributions. 



